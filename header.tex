\usepackage[dvipsnames]{xcolor}
\usepackage{tikz}
\usetikzlibrary{calc}
\usepackage{enumitem}
\usepackage{alltt}
\usepackage{amsfonts}
\usepackage{amsmath}
\usepackage{amssymb}
\usepackage{amsthm}
\usepackage{booktabs}
\usepackage{bm}
\usepackage{bbm}
\usepackage{caption}
\usepackage{graphicx}
\usepackage{mathrsfs}
\usepackage{mathdots}
\usepackage{mathtools}
\usepackage{microtype}
\usepackage{multirow}
\usepackage{soul}
\usepackage{empheq}
\usepackage{mdframed}

% mdframed environments remain unchanged:
\newmdenv[
  innerbottommargin = 4mm,
  middlelinewidth = 0.3mm,
  linecolor = darkgray,
  backgroundcolor=TealBlue!10,
  nobreak=true
]{boxexample}
\newenvironment{ex}{\boxexample\begin{eg}}{\end{eg}\endboxexample}

\newmdenv[
  innerbottommargin = 4mm,
  middlelinewidth = 0.3mm,
  linecolor = darkgray,
  backgroundcolor=Salmon!10,
  nobreak=true
]{boxtheo}

\newmdenv[
  innerbottommargin = 4mm,
  middlelinewidth = 0.3mm,
  linecolor = darkgray,
  backgroundcolor=Goldenrod!20,
  nobreak=true
]{boxdefinition}
\newenvironment{thm}{\boxtheo\begin{theo}}{\end{theo}\endboxtheo}
\newenvironment{boxdef}{\boxdefinition\begin{defi}}{\end{defi}\endboxtheo}

\newenvironment{enum}{\newblock\begin{enumerate}[label=(\alph*)]}{\end{enumerate}}
\newenvironment{statement}[1]{\smallskip\noindent\color{BrickRed} {\bf #1.}}{}
\newenvironment{prob}{\color{BrickRed}\begin{probinner}}{\end{probinner}}
\newenvironment{comments}{\color{BrickRed}\begin{commentinner}}{\end{commentinner}}
\usepackage{imakeidx}
\newcommand{\argmin}{\operatornamewithlimits{arg\,min}}
\newcommand{\argmax}{\operatornamewithlimits{arg\,max}}

\makeindex[intoc, title=Index]
\usepackage[pdftex,
  hidelinks,
  pdfauthor={Kadin Zhang},
  pdfsubject={},
  pdftitle={},
  pdfkeywords={}]{hyperref}
\renewcommand\printindex{}

% Set standard 1-inch margins using the geometry package.
\usepackage[margin=1in]{geometry}

% Theorem styles and macros:
\newtheoremstyle{definition}{}{}{}{}{\bfseries}{:}{.5em}{\thmname{#1}\thmnumber{ #2}\thmnote{ (\textcolor{darkgray}{#3})}}
\theoremstyle{definition}

\newtheorem*{aim}{Aim}
\newtheorem*{axiom}{Axiom}
\newtheorem*{claim}{Claim}
\newtheorem*{cor}{Corollary}
\newtheorem*{conjecture}{Conjecture}
\newtheorem*{commentinner}{Comments}
\newtheorem*{defi}{Definition}
\newtheorem*{eg}{Example}
\newtheorem*{exc}{Exercise}
\newtheorem*{fact}{Fact}
\newtheorem*{law}{Law}
\newtheorem*{lemma}{Lemma}
\newtheorem*{notation}{Notation}
\newtheorem*{prop}{Proposition}
\newtheorem*{question}{Question}
\newtheorem*{rrule}{Rule}
\newtheorem*{theo}{Theorem}
\newtheorem*{assumption}{Assumption}

\newtheorem*{remark}{Remark}
\newtheorem*{warning}{Warning}
\newtheorem*{exercise}{Exercise}

\newtheorem{nthm}{Theorem}[section]
\newtheorem{nlemma}[nthm]{Lemma}
\newtheorem{nprop}[nthm]{Proposition}
\newtheorem{ncor}[nthm]{Corollary}
\newtheorem{probinner}[nthm]{Problem}

\renewcommand{\labelitemi}{--}
\renewcommand{\labelitemii}{$\circ$}
\renewcommand{\labelenumi}{(\roman{*})}

\newcommand\qedsym{\hfill\ensuremath{\square}}
\def\st{\bgroup \ULdepth=-.55ex \ULset}

% Mathematics symbols and operator definitions:
\newcommand{\leb}{\text{Leb}}

\newcommand{\C}{\mathbb{C}}
\newcommand{\CP}{\mathbb{CP}}
\newcommand{\GG}{\mathbb{G}}
\newcommand{\N}{\mathbb{N}}
\newcommand{\F}{\mathbb{F}}
\newcommand{\Q}{\mathbb{Q}}
\newcommand{\R}{\mathbb{R}}
\newcommand{\RP}{\mathbb{RP}}
\newcommand{\T}{\mathbb{T}}
\newcommand{\Z}{\mathbb{Z}}
\renewcommand{\H}{\mathbb{H}}

\DeclarePairedDelimiter\parens{\lparen}{\rparen}
\DeclarePairedDelimiter\abs{\lvert}{\rvert}
\DeclarePairedDelimiter\norm{\lVert}{\rVert}
\DeclarePairedDelimiter\floor{\lfloor}{\rfloor}
\DeclarePairedDelimiter\ceil{\lceil}{\rceil}
\DeclarePairedDelimiter\braces{\lbrace}{\rbrace}
\DeclarePairedDelimiter\bracks{\lbrack}{\rbrack}
\DeclarePairedDelimiter\angles{\langle}{\rangle}

\DeclarePairedDelimiter\bigp{\Big(}{\Big)}
\DeclarePairedDelimiter\bigc{\Bigg\{}{\Bigg\}}
\DeclarePairedDelimiter\biga{\Bigg|}{\Bigg|}

\DeclareMathOperator{\poly}{poly}
\DeclareMathOperator{\polylog}{polylog}
\DeclareMathOperator{\size}{size}
\DeclareMathOperator{\sgn}{sgn}
\DeclareMathOperator{\dist}{dist}
\DeclareMathOperator{\vol}{vol}
\DeclareMathOperator{\spn}{span}
\DeclareMathOperator{\supp}{supp}
\DeclareMathOperator{\tr}{tr}
\DeclareMathOperator{\Tr}{Tr}
\DeclareMathOperator{\codim}{codim}
\DeclareMathOperator{\diag}{diag}

\DeclareMathOperator{\lcm}{lcm}
\DeclareMathOperator{\OPT}{OPT}
\DeclareMathOperator{\DFT}{DFT}
\DeclareMathOperator{\rank}{rank}
\DeclareMathOperator{\nul}{nul}
\DeclareMathOperator{\ord}{ord}
\DeclareMathOperator{\diam}{diam}
\DeclareMathOperator{\erf}{erf}
\DeclareMathOperator{\err}{err}
\newcommand{\eps}{\varepsilon}

\DeclareMathOperator{\adj}{adj}
\DeclareMathOperator{\Aut}{Aut}
\DeclareMathOperator{\Char}{char}
\DeclareMathOperator{\Hom}{Hom}
\DeclareMathOperator{\id}{id}
\DeclareMathOperator{\image}{image}
\DeclareMathOperator{\im}{im}
\newcommand{\Frob}{\mathrm{Frob}}

\newcommand{\bolds}[1]{{\bfseries #1}}
\newcommand{\cat}[1]{\mathsf{#1}}
\newcommand{\mc}[1]{\mathcal{#1}}
\newcommand{\ph}{\,\cdot\,}
\newcommand{\term}[1]{\emph{#1}\index{#1}}
\newcommand{\phantomeq}{\hphantom{{}={}}}

\DeclareMathOperator{\PoA}{PoA}
\DeclareMathOperator{\PoS}{PoS}
\DeclareMathOperator{\Ber}{Ber}
\DeclareMathOperator{\io}{i.o.}
\DeclareMathOperator{\ev}{ev}
\DeclareMathOperator{\U}{U}
\DeclareMathOperator{\betaD}{beta}
\DeclareMathOperator{\bias}{bias}
\DeclareMathOperator{\Bin}{Bin}
\DeclareMathOperator{\corr}{corr}
\DeclareMathOperator{\cov}{cov}
\DeclareMathOperator{\var}{var}
\DeclareMathOperator{\gammaD}{gamma}
\DeclareMathOperator{\mse}{mse}
\DeclareMathOperator{\multinomial}{multinomial}
\DeclareMathOperator{\Poisson}{Poisson}
\newcommand{\E}{\mathbf{E}}
\newcommand{\wto}{\rightharpoonup}
\newcommand{\dto}{\xrightarrow{\text{d}}}
\newcommand{\pto}{\xrightarrow{\text{p}}}
\newcommand{\ato}{\xrightarrow{\text{a.s.}}}

\let\P\relax
\newcommand{\P}{\mathbf{P}}
\let\Pr\relax
\DeclareMathOperator*{\Pr}{\mathbf{Pr}}

\let\Im\relax
\let\Re\relax

\makeatother
