\documentclass{article}

\title{Condensed intuitions for Lebesgue integration}
\author{Kadin Zhang}
\date{6/24}

\usepackage[dvipsnames]{xcolor}
\usepackage{tikz}
\usetikzlibrary{calc}
\usepackage{enumitem}
\usepackage{alltt}
\usepackage{amsfonts}
\usepackage{amsmath}
\usepackage{amssymb}
\usepackage{amsthm}
\usepackage{booktabs}
\usepackage{bm}
\usepackage{bbm}
\usepackage{caption}
\usepackage{graphicx}
\usepackage{mathrsfs}
\usepackage{mathdots}
\usepackage{mathtools}
\usepackage{microtype}
\usepackage{multirow}
\usepackage{soul}
\usepackage{empheq}
\usepackage{mdframed}

% mdframed environments remain unchanged:
\newmdenv[
  innerbottommargin = 4mm,
  middlelinewidth = 0.3mm,
  linecolor = darkgray,
  backgroundcolor=TealBlue!10,
  nobreak=true
]{boxexample}
\newenvironment{ex}{\boxexample\begin{eg}}{\end{eg}\endboxexample}

\newmdenv[
  innerbottommargin = 4mm,
  middlelinewidth = 0.3mm,
  linecolor = darkgray,
  backgroundcolor=Salmon!10,
  nobreak=true
]{boxtheo}

\newmdenv[
  innerbottommargin = 4mm,
  middlelinewidth = 0.3mm,
  linecolor = darkgray,
  backgroundcolor=Goldenrod!20,
  nobreak=true
]{boxdefinition}
\newenvironment{thm}{\boxtheo\begin{theo}}{\end{theo}\endboxtheo}
\newenvironment{boxdef}{\boxdefinition\begin{defi}}{\end{defi}\endboxtheo}

\newenvironment{enum}{\newblock\begin{enumerate}[label=(\alph*)]}{\end{enumerate}}
\newenvironment{statement}[1]{\smallskip\noindent\color{BrickRed} {\bf #1.}}{}
\newenvironment{prob}{\color{BrickRed}\begin{probinner}}{\end{probinner}}
\newenvironment{comments}{\color{BrickRed}\begin{commentinner}}{\end{commentinner}}
\usepackage{imakeidx}
\newcommand{\argmin}{\operatornamewithlimits{arg\,min}}
\newcommand{\argmax}{\operatornamewithlimits{arg\,max}}

\makeindex[intoc, title=Index]
\usepackage[pdftex,
  hidelinks,
  pdfauthor={Kadin Zhang},
  pdfsubject={},
  pdftitle={},
  pdfkeywords={}]{hyperref}
\renewcommand\printindex{}

% Set standard 1-inch margins using the geometry package.
\usepackage[margin=1in]{geometry}

% Theorem styles and macros:
\newtheoremstyle{definition}{}{}{}{}{\bfseries}{:}{.5em}{\thmname{#1}\thmnumber{ #2}\thmnote{ (\textcolor{darkgray}{#3})}}
\theoremstyle{definition}

\newtheorem*{aim}{Aim}
\newtheorem*{axiom}{Axiom}
\newtheorem*{claim}{Claim}
\newtheorem*{cor}{Corollary}
\newtheorem*{conjecture}{Conjecture}
\newtheorem*{commentinner}{Comments}
\newtheorem*{defi}{Definition}
\newtheorem*{eg}{Example}
\newtheorem*{exc}{Exercise}
\newtheorem*{fact}{Fact}
\newtheorem*{law}{Law}
\newtheorem*{lemma}{Lemma}
\newtheorem*{notation}{Notation}
\newtheorem*{prop}{Proposition}
\newtheorem*{question}{Question}
\newtheorem*{rrule}{Rule}
\newtheorem*{theo}{Theorem}
\newtheorem*{assumption}{Assumption}

\newtheorem*{remark}{Remark}
\newtheorem*{warning}{Warning}
\newtheorem*{exercise}{Exercise}

\newtheorem{nthm}{Theorem}[section]
\newtheorem{nlemma}[nthm]{Lemma}
\newtheorem{nprop}[nthm]{Proposition}
\newtheorem{ncor}[nthm]{Corollary}
\newtheorem{probinner}[nthm]{Problem}

\renewcommand{\labelitemi}{--}
\renewcommand{\labelitemii}{$\circ$}
\renewcommand{\labelenumi}{(\roman{*})}

\newcommand\qedsym{\hfill\ensuremath{\square}}
\def\st{\bgroup \ULdepth=-.55ex \ULset}

% Mathematics symbols and operator definitions:
\newcommand{\leb}{\text{Leb}}

\newcommand{\C}{\mathbb{C}}
\newcommand{\CP}{\mathbb{CP}}
\newcommand{\GG}{\mathbb{G}}
\newcommand{\N}{\mathbb{N}}
\newcommand{\F}{\mathbb{F}}
\newcommand{\Q}{\mathbb{Q}}
\newcommand{\R}{\mathbb{R}}
\newcommand{\RP}{\mathbb{RP}}
\newcommand{\T}{\mathbb{T}}
\newcommand{\Z}{\mathbb{Z}}
\renewcommand{\H}{\mathbb{H}}

\DeclarePairedDelimiter\parens{\lparen}{\rparen}
\DeclarePairedDelimiter\abs{\lvert}{\rvert}
\DeclarePairedDelimiter\norm{\lVert}{\rVert}
\DeclarePairedDelimiter\floor{\lfloor}{\rfloor}
\DeclarePairedDelimiter\ceil{\lceil}{\rceil}
\DeclarePairedDelimiter\braces{\lbrace}{\rbrace}
\DeclarePairedDelimiter\bracks{\lbrack}{\rbrack}
\DeclarePairedDelimiter\angles{\langle}{\rangle}

\DeclarePairedDelimiter\bigp{\Big(}{\Big)}
\DeclarePairedDelimiter\bigc{\Bigg\{}{\Bigg\}}
\DeclarePairedDelimiter\biga{\Bigg|}{\Bigg|}

\DeclareMathOperator{\poly}{poly}
\DeclareMathOperator{\polylog}{polylog}
\DeclareMathOperator{\size}{size}
\DeclareMathOperator{\sgn}{sgn}
\DeclareMathOperator{\dist}{dist}
\DeclareMathOperator{\vol}{vol}
\DeclareMathOperator{\spn}{span}
\DeclareMathOperator{\supp}{supp}
\DeclareMathOperator{\tr}{tr}
\DeclareMathOperator{\Tr}{Tr}
\DeclareMathOperator{\codim}{codim}
\DeclareMathOperator{\diag}{diag}

\DeclareMathOperator{\lcm}{lcm}
\DeclareMathOperator{\OPT}{OPT}
\DeclareMathOperator{\DFT}{DFT}
\DeclareMathOperator{\rank}{rank}
\DeclareMathOperator{\nul}{nul}
\DeclareMathOperator{\ord}{ord}
\DeclareMathOperator{\diam}{diam}
\DeclareMathOperator{\erf}{erf}
\DeclareMathOperator{\err}{err}
\newcommand{\eps}{\varepsilon}

\DeclareMathOperator{\adj}{adj}
\DeclareMathOperator{\Aut}{Aut}
\DeclareMathOperator{\Char}{char}
\DeclareMathOperator{\Hom}{Hom}
\DeclareMathOperator{\id}{id}
\DeclareMathOperator{\image}{image}
\DeclareMathOperator{\im}{im}
\newcommand{\Frob}{\mathrm{Frob}}

\newcommand{\bolds}[1]{{\bfseries #1}}
\newcommand{\cat}[1]{\mathsf{#1}}
\newcommand{\mc}[1]{\mathcal{#1}}
\newcommand{\ph}{\,\cdot\,}
\newcommand{\term}[1]{\emph{#1}\index{#1}}
\newcommand{\phantomeq}{\hphantom{{}={}}}

\DeclareMathOperator{\PoA}{PoA}
\DeclareMathOperator{\PoS}{PoS}
\DeclareMathOperator{\Ber}{Ber}
\DeclareMathOperator{\io}{i.o.}
\DeclareMathOperator{\ev}{ev}
\DeclareMathOperator{\U}{U}
\DeclareMathOperator{\betaD}{beta}
\DeclareMathOperator{\bias}{bias}
\DeclareMathOperator{\Bin}{Bin}
\DeclareMathOperator{\corr}{corr}
\DeclareMathOperator{\cov}{cov}
\DeclareMathOperator{\var}{var}
\DeclareMathOperator{\gammaD}{gamma}
\DeclareMathOperator{\mse}{mse}
\DeclareMathOperator{\multinomial}{multinomial}
\DeclareMathOperator{\Poisson}{Poisson}
\newcommand{\E}{\mathbf{E}}
\newcommand{\wto}{\rightharpoonup}
\newcommand{\dto}{\xrightarrow{\text{d}}}
\newcommand{\pto}{\xrightarrow{\text{p}}}
\newcommand{\ato}{\xrightarrow{\text{a.s.}}}

\let\P\relax
\newcommand{\P}{\mathbf{P}}
\let\Pr\relax
\DeclareMathOperator*{\Pr}{\mathbf{Pr}}

\let\Im\relax
\let\Re\relax

\makeatother


\begin{document}
\maketitle
{
\small
\setlength{\parindent}{0em}
\setlength{\parskip}{1em}
}

\section{Motivation}
Consider the Riemann integral
\[
    \int_{0}^{1} \frac{1}{\sqrt{x} } \,dx. 
\]
Recall that Riemann integrals are defined where the upper Riemann integral and lower Riemann integral (that is, the infimum over ``overestimates'' and supremum over ``underestimates'' defined by a partitioning of the interval of integration) coincide. But, no matter how we partition $[0, 1]$, the function over the first interval will always have no finite supremum. This contradicts the intuition that we should have 
\[
    \int_{0}^{1} \frac{1}{\sqrt{x} } \,dx = \lim_{a \to 0^+ } \int_{a}^{1} \frac{1}{\sqrt{x} } \,dx  = 2.  
\]
Consider also integrating the function 
\[
    f(x) = \begin{dcases}
        1, &\text{ if } x \in \Q  ;\\
        0, &\text{ otherwise} .
    \end{dcases}
\]
over $[0, 1]$. The infimum over any subinterval is $0$, while the supremum is $1$. The overarching problem is that this partitioning method is too restrictive. We would ideally like to define the integral over arbitary disjoint partition of ``nice sets'', where we have a notion of ``measure'' on these sets. Then, for the last example, we could have 
\[
    [0, 1] = \braces*{x \in [0, 1] : x \in \Q } \cap  \braces*{x \in [0, 1] : x \notin  \Q },
\]
where the left side of the intersection should have measure $0$, making the integral $0$. Now, we just have to define these sets and this measure! 

\section{Constructing a measure}
Let $X$ be an abstract space, and $m$ be a collection of subsets of $X$. We only ask that a measure satisfy two properties, namely that the measure of the empty set is $0$, and the measure of a countable disjoint union is the sum of the measures of each set in the union. 
\begin{defi}[Measure]
    $\mu : (X, m) \to  [0, \infty ]$ is a \emph{measure} if 
    \begin{enum}
        \item $\mu (\varnothing ) = 0$
        \item $\mu \biggl( \bigcup_n E_n  \biggr) = \sum_{n} \mu (E_n)  $ for disjoint  $E_1 , E_2 , \dots \in m$. 
    \end{enum}
\end{defi}

Beginning with some space $X$, we will construct this measure by first defining an \emph{outer measure} that operates on the powerset of $X$, then defining the set $m$ which will make countable additivity hold. But first, to even reason about measures, we need to ensure that the collection of subsets that are measured are well behaved:
\begin{defi}[$\sigma$-algebra]
    A collection of sets $m$ is said to be a $\sigma$-algebra if 
    \begin{enum}
        \item $\varnothing \in m$.  
        \item $E \in m \implies  X \setminus  E \in M$. 
        \item $(E_n) \in m \implies \bigcup_n E_n \in m$. 
    \end{enum}
\end{defi}

\begin{defi}[Outer measure]
    Let $\mc{E} \subseteq \mc{P} (X)$ be a family of elementary sets. We require $\varnothing \in \mc{E} $ and $(X_n ) \subset \mc{E} $, where $X = \bigcup_n  X_n $. We also define a function $\rho  : \mc{E} \to [0, \infty ]$ with $\rho (\varnothing ) = 0$. Then define the outer measure as:
\[
    \mu^*(E) = \inf \left\{ \sum_{n} \rho (E_n ) : E_n \in \mc{E} , E \subseteq \bigcup_n E_n  \right\} . 
\]
\end{defi}

Essentially we are approximating any subset $E\subseteq X $ through countably many elementary ``outer'' sets for which a measure is easy to define. In particular, we construct the \emph{Lebesgue outer measure} on $\R^n $ by choosing $\mc{E} $ to be the set of rectangles. 

Finally, we will take the collection of subsets $m$ to be the set of \emph{$\mu ^* $-measurable sets}: 
\begin{defi}[$\mu^* $-measurable]
    A set $E\subseteq X$ is said to be $\mu^* $\emph{-measurable} if for all $F\subseteq X$,
    \[
        \mu^* (F) = \mu^* (F\cap E) + \mu^* (F\setminus E). 
    \]
    We call this set $m^* $. Intuitively, $E$ splits sets $F$ ``cleanly'' into $F \cap E$ and $F \setminus  E$. 
\end{defi}
That $\mu ^* $ restricted to $m^* $ is a measure and $m^* $ is a \emph{$\sigma $-algebra} is a result of Caratheodory's theorem. 

\section{Integration}
Having defined measures and measurable sets, now we can reason about measure spaces $(X, m, \mu )$ and functions between these spaces. Measurable functions are mappings between measurable spaces that respect the $\sigma$-algebras. These are the functions we will integrate over, since we can show that measurable functions are the increasing limit of simple functions. 
\begin{defi}[Measurable functions]
    Let $X, Y$ be nonempty sets, $m, n$ $\sigma$-algebras on $X, Y$ respectively. Then, $f : X\to Y$ is said to be \emph{measurable} if for all $F \in n$,
    \[
        f^{-1} (F) \in m. 
    \]
    If $Y = \R^N $, $n$ is assumed to be $\mc{B} (\R^N )$. If further $X = \R^d $, we classify measurable functions into two categories (where the latter is assumed when not supplied): 
    \begin{enum}
        \item \emph{Lebesgue measurable}: $B \in \mc{B} (\R^N )$ implies $f^{-1} (B) \in m^*$. 
        \item \emph{Borel measurable}: $B \in \mc{B} (\R^N )$ implies $f^{-1} (B) \in \mc{B} (\R^d )$.  
    \end{enum}
\end{defi}

Now we can define the Lebesgue integral. As we wanted, this now considers arbitrary partitions of the domain $X$ into measurable sets. 
\begin{defi}[Lower Lebesgue sum]
    Let $P = \{ A_1 , \dots, A_m  \} $ be a partition of $X$ with $A_i \in m$. The \emph{lower Lebesgue sum} with respect to $P$ is defined by 
   \[
       L(f, P) \coloneqq \sum_{i}^m  \mu (A_i ) \inf _{A_i } f.  
   \]
\end{defi}

\begin{defi}[Lebesgue integral]
   \[
       \int_X f d \mu \coloneqq \sup_P L(f, P)  . 
   \]
\end{defi}

Now, we reap the rewards of the Lebesgue integral:  
\begin{thm}[Monotone Convergence Theorem]
    Let $(X, m, \mu )$ and consider a sequence of measurable functions 
    \[
        0\leq  f_1  \leq  \dots \leq  f_n  \to f
    \]
    for some $f : X\to [0, \infty ]$. Then, 
    \[
        \int _X f d \mu = \lim_{k \to \infty} \int _X  f_k d \mu . 
    \]
\end{thm}

\begin{thm}[Fatou's Lemma]
    Let $(X,m,\mu)$ and $f_n : X\to [0,\infty ]$ measurable.
    \[
        \int \liminf_n f_n d\mu \leq \liminf_n \int f_n  d\mu . 
    \]
\end{thm}

\begin{thm}[Dominated convergence theorem]
    Let $(X,m,\mu)$, $f_n \to \overline{\R} $ measurable functions with limit $f$ $\mu$ a.e. Assume there exists an integrable function $g : X \to [0,\infty]$ such that $| f_n (x) |\leq g(x) $ $\mu $ a.e. Then $f$ is Lebesgue integrable and 
    \[
        \int _X f_n d\mu \to \int _X f d\mu. 
    \]
\end{thm}



\end{document}